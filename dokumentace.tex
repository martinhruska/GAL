\documentclass[a4paper, 12pt]{article}
\usepackage[left=1.5cm, text={18cm, 25cm}, top=2.5cm]{geometry}
\usepackage[utf8]{inputenc}
\usepackage[czech]{babel}
\usepackage{cite}
\usepackage{graphicx}
\usepackage{float}
\usepackage{amsmath}
\usepackage{amssymb}
\usepackage{tikz}
\usepackage{url}
\usepackage{comment}
\usepackage[longend,ruled,vlined,commentsnumbered,linesnumbered]{algorithm2e}
\newcommand{\myuv}[1]{\quotedblbase #1\textquotedblleft}

\title{Simulation of non-deterministic finite tree automata}
\author{Martin Hruška, Petr Šebek\\\{xhrusk16, xsebek02\}@stud.fit.vutbr.cz}

\date{}
\begin{document}

\maketitle

\section{Introduction}
\label{sec:intro}

Non-deterministic finite tree automata (NTA) is formalism often used in the field of formal verification.
For example, it can be used for shape analysis of programs manipulating complex data structure where
state of heap is represented by a set of the tree automata [cite forester].
The mentioned technqiue is based on the framework of abstract interpretation where one of fundamentals
operations is computing fixpoint (mainly for the cycles in a program) what includes checking inclusion of
tree automata languages.
Checking inclusion is easy for deterministict tree automata (DTA) but much harder for NTA because
naive algorithm has exponential compexity.
However, one can make the algorithms for checking inclusion more efficient [cite antichains] by computing simulation relation over the states of checked automata.
Please note, that it is not possbile to use DTA in these techniques because NTA can represent same language in much more
conciser way and determinization could lead to exponental grow of the states.

Being more general, simulation relation is attribute of graphs [cite Herzinger], not only automata.
In case of tree automata it is possible to compute upward or downward simulation but in this
work we consider only downward simulation.
As we mentioned earlier simulation brings reduction of states which we need explore performing
certain algorithms over tree automata because it often holds that when state $p$ is simulated by $q$
we don't to further explore state $p$.
Moreover, it is possible also to perform reduction of states of a NFA by computing
equivalence classes on simulation relation and merging states in a same class.

Another technique dealing with complexity of NFA is their representation using mutli-terminal binary decision diagrams (MTBDD).
The symbols of a NFA are encoded to binary representation so the transitions can represented by shared MTBDD what is very efficient mainly for NFA with large alphabeth
(in the second possible representation -- symbols are represented explicitly for each transition).
A~NFA could be represented in bottom-up way or top-down way by MTBDD, in this work we consider only top-down way.

In this work we would like to combine straightwordness of implementation of simulation over explicitly
represented tree automata and conciser representation of automata by MTBDD.
We suppose NTA represented by MTBDD and we will compute simulation by it conversion
to explicit one with gradual reduction of number of the NFA symbols.
We use computed simulation for more efficient inclusion computation and evaluate whether the method brings any advantage.
The implementation is realized as an extension of VATA library which state-of-the-art library for NTA manipulation.

In Section \ref{sec:analysis} we give formal definitions, in Section \ref{sec:bdd} MTBDD representation of automata is described.
Section \ref{sec:bdd} describes VATA library, Section \ref{sec:vata} provides description of design of our solution.
Implementation details take place in Section \ref{sec:impl} and finally experiments are evaluated in Section \ref{sec:exps}.
\cite{autor:yes}

\section{Preliminaries}
\label{sec:analysis}
In this section NTA and simulation over NTA states will be defined more formally.

A~\emph{ranked alphabeth} is a finite set of symbols $\Sigma$ associated with a mapping $\#: \Sigma \rightarrow \mathbb{N_0}$
that assigns ranks to symbols. A~\emph{tree} is a graph $t$ which is either empty or it has exactly one root and each of its
nodes is the $i$-th successor of at most one node $v$ for some $i \in \mathbb{N_0}$

A~\emph{finite, non-deterministic, top-down tree automata} is a quadratuple $A=(Q, \Sigma, \delta, R)$ where
$Q$ is a finite set of \emph{states}, $R\subseteq Q$ is a set of \emph{root states}, $\Sigma$ is a ranked alphabeth,
$\delta$ is a set of the transition rules.
Each transition is a triple of the form $(q,a,q_1, \ldots, q_n)$ where $n \geq 0$, $q, q_0 \ldots q_n \in Q$, $a \in \Sigma$ and $\#(a) = n$.
When $n = 0$ then such a transition is called a \emph{leaf rules}.
A~\emph{bottom-up} automaton is a quadratuple $B=(Q, \Sigma, \delta, F)$, where $Q$, $F$ are same as for $top-down$ automaton, $F\subseteq Q$
is a set of final states and $\delta$ is a transition relation with rules which are in the form $(q_1,\ldots, q_n,a,q)$ where $n \geq 0$ and $\#(a) = n$.
We can interchange the two notations because non-deterministic automata are known to have same expressive power in bottom-up and top-down version. [cite somtheing]

[TODO: maybe define semantics of TA]


For a bottom-up NTA $A=(Q, \Sigma, \delta, R)$, a \emph{downward simulation} $\preceq\, \subseteq Q\times Q$ is binary relation such that $q \preceq p$
and $(q,a,q_1,\ldots, q_n),a,q)$ then $\exists (p_1, \ldots, p_n) \rightarrow p \wedge \forall i \in {1 \ldots n}: q_i \preceq p_i$.

\section{BDD representation}
\label{sec:bdd}

how the tree automata is represented

\section{VATA}
\label{sec:vata}
VATA is library for efficient manipulation with NTA [cite VATA].
It provides both encoding for NTA -- semi-symbolic via MTBDD and explicit.
VATA is open-source licensed by GPL license and it is written in C++.
It currently supports TIMBUK as input format.

The main goal of VATA is to provide state-of-the-art algorithms for inclusion checking
but it also contains implementation of standard operations like union or intersection.
As we mentioned in Section \ref{sec:intro} inclusion checking is related to simulation
since simulation relation could bring more efficiency to the state-of-the-art algorithms like [cite antichians].
There are already implemented algorithms for computing simulation in VATA but not one of them uses
method which we used.
The algorithms for simulation computing in VATA uses encoding to LTL systems [cite tacas08] in case of explicit encoding
and there are some implementations over MTBDD which are not known to work at all.

\section{Design}
\label{sec:design}

This section describes conceptual design of our method for computing simulation and related algorithms.
For implementation design please see Section \ref{sec:impl}.

As the input of our method we take a NTA represented by MTBDD in top-down way
which we translate into a NTA represented explicitly and we also reduce symbols.
This done by following general method doing reduction of alphabeth and yielding a new NTA.
Let have a NFA $A=(Q, \Sigma, \delta, F)$, a new ranking alphabeth $\Sigma'$ and mapping $f: 2^\Sigma \rightarrow \Sigma'$,
when there is $((a_1(q_1,\ldots,q_n) \rightarrow q) \in \delta \wedge \ldots \wedge (a_n(q_1,\ldots,q_n) \rightarrow q) \in \delta) \wedge
((a_1(r_1,\ldots,r_n) \rightarrow q) \in \delta \wedge \ldots \wedge (a_n(r_1,\ldots,r_n) \rightarrow q) \in \delta)$
such that $a_i \neq a_{i+1}$ for any $i\in \{1..n\}$ then $(\{a_1, \ldots, a_n\}) \rightarrow A) \in f$.
It is also constructed a new NFA $A' = (Q, \Sigma', \delta', F)$ where $Q, F$ are unchanged to $A$ 
and $\Sigma'$ is the new ranked alphabeth obtained by mapping $f$ and $\delta'$ is transition rules set such that
$(a_1(q_1, \ldots, q_n) \rightarrow q) \in \delta \wedge \ldots \wedge (a_n(q_1, \ldots, q_n) \rightarrow q) \in \delta$ and
$ \{a_1,\ldots,a_n\}  \rightarrow A~\in f$ for some $A \in \Sigma'$, then $A(q_1, \ldots, q_n) \rightarrow q) \in \delta$
for all possible tuples $q_1,\ldots,q_n$ [TODO which tuples].
In our case, the input NFA is semi-symbolic represented one and the output NTA is the same automata in explicit representation.
The change of NTA representation is done in level of implementation details and is not directly related to a design of generic algorithm.
The whole method of translation is described again in Algortihm \ref{alg:translate} to provide more easy to follow way of method description.

\begin{algorithm}
\label{alg:translate}
\KwIn{NTA $A=(Q,\Sigma, \delta, F)$}
    \KwOut{NTA $A'=(Q,\Sigma', \delta', F)$}
    $\delta' = \emptyset $\;
    $\Sigma' = \emptyset $\;
	$f = \emptyset$\;
	\ForEach{$q \in Q$}
    {
		\ForEach{$(q_1,\ldots, q_n)$, such that $(a(q_1, \ldots, q_n)\rightarrow q) \in \delta)$ for any $a\in \Sigma$}
		{
			$S_q = \{ a\in \Sigma \,|\, \exists(a(q_1,\ldots,q_n) \rightarrow q) \in \delta\}$\;
			\ForEach{$(r_1,\ldots, r_n)$, such that $(b(r_1, \ldots, r_n)\rightarrow q) \in \delta)$ for any $a\in \Sigma$}
			{
				$S_r = \{a\in \Sigma \,|\, \exists\, (a(r_1,\ldots,r_n) \rightarrow q) \in \delta\}$\;
				\If{$(S_q \cap S_r \rightarrow A') \not\in f$ for any $A' \in \Sigma'$}
				{
					add $A'$ to $\Sigma'$ such that $A' \not\in \Sigma'$\;
					add $(S_q \cap S_r \rightarrow A')$ to $f$\;
				}
				add $f(S_q \cap S_r)(q_1,\ldots,) \rightarrow q$ to $\delta'$\;
			}
		}
    }
\caption{NTA symbol reduction yielding a new NTA}
\end{algorithm}

[TODO]
Finally, we compute downward simulation over explicit NTA by Algorithm \ref{alg:sim}.

\section{Implementation}
\label{sec:impl}

As we mentioned in Section \ref{sec:vata} VATA library already has infrastructure needed for implementation of
our method.
It contains support for parsing input NTA, its semi-symbolic and explicit representation and implemented
state-of-the-art algorithm for inclusion checking.
So we decided to implement our solution as extension of VATA library.
We design our implementation like a stand-alone module which is compiled seperately to the rest of library (using CMake building system exploited in VATA).
Module needs just include some headers for knowing interface of VATA classes for NTA representation.

When we focuse on implementation of our module itself we use following classes:
\begin{itemize}
	\item \emph{BDDTopDownSimExpl} Class translates MTBDD represented automaton to explicit one with symbol translation
	\item \emph{BDDBDDTopDownSimComputer} Class computes downward simulation over explicit tree automata
\end{itemize}

We exploited following classes from VATA implementation:

\begin{itemize}
	\item \emph{BDDTDTreeAutCore} Class for semi-symbolic representation of NTA using MTBDD.
	\item \emph{ExplicitTreeAutCore} Class for explicit representation of NTA.
	\item \emph{BinaryDiscontRelation} Class for representation of simulation relation.
\end{itemize}

\subsection{ExplicitRepresentation}

\subsection{Semi-Symbolic Representation}

\section{Experiments}
\label{sec:exps}

inclusion with our simulation is much faster

\section{Conclusion}
\label{sec:end}

This works describes implementation of a method for computing simulation over NFA and its implementation.
Our method takes NTA represented in semi-symbolic top-down way using MTBDD, it
turns it to explicit representation and compute downward simulation over it.

We implemented our method as module to VATA library exploiting its infrastructure for manpilationg NTA.
The method has been evaluated on set of NTA and it has shown that our method and found that it is very awsome.
It 100000x faster then inclusion checking without simulation

Further research may be more efficient manipulation with MTBDD during traslation to explicit representation or
extending it to bottom-up automata.
It is also possible to implement NFA reduction which uses computed simulation.

\newpage
\appendix
\section{User guide}
\label{app:usage}

for installation please read README
for run, please read README-gal


\bibliography{literatura}
\bibliographystyle{czechiso}
\end{document}
